\documentclass[preprint, letterpaper, nobibnotes, aps, superscriptaddress,prb]{revtex4-1}

\usepackage{amsmath,amssymb} 
\usepackage{bm}
\usepackage{hyperref}
\usepackage{subfigure}
\usepackage{graphicx}
\usepackage{color}
\usepackage{float}
\usepackage{multirow}
\usepackage{etoolbox}
\usepackage{booktabs}
\usepackage{ltablex}
\usepackage{xr}
\begin{document}


\section{Bending of elastic fibres in viscous flows: the influence of confinement}

\textit{2013. JFM}

\subsection{Summary}
Presents series of microfluidic experiments examining the flow of a viscous fluid past an elastic fibre in a three-dimensional channel.

Their experiments show that there is a linear relationship
between deflection and flow rate for highly confined fibres at low flow rates, which
inspires an asymptotic treatment of the problem in this regime. 

The analyze the competing effects of flexion and leakage.
\begin{itemize}
\item
Schematic of the experimental setup
\begin{figure}[H]
  \centering \resizebox{0.9\textwidth}{!}{\includegraphics{Wexler13_1.png}}
  \caption{}\label{fig1}
\end{figure}
  \begin{figure}[H]
  \centering \resizebox{0.9\textwidth}{!}{\includegraphics{Wexler13_2.png}}
  \caption{}\label{fig2}
\end{figure}
\item
Parameters space variables: 
\begin{enumerate}
\item
$u$(Fibre tip deflection)
\item 
$Re = 500$
\item 
Permeability parameter (density):
\begin{align*}
\beta=\Bigg(\frac{(D-d)^3}{4wD^3}\Bigg)H
\end{align*}
\item 
Bending parameter:
\begin{align*}
\epsilon=\Big(\frac{D+d}{2EI}\Big)\Big(\frac{12\mu Q}{D^3}\Big)H^3
\end{align*}
\end{enumerate}

\item
Results: 
\begin{itemize}
	\item
	
	\item Fibre tip deflection ($u$) with flow rate ($Q$).
  \begin{figure}[H]
  \centering \resizebox{0.9\textwidth}{!}{\includegraphics{Wexler13_3.png}}
  \caption{}\label{fig3}
\end{figure}

	\item Constant pressure curves and streamlines for different $c$ ($c=\frac{h}{H}$).
  \begin{figure}[H]
  \centering \resizebox{0.9\textwidth}{!}{\includegraphics{Wexler13_4.png}}
  \caption{}\label{fig4}
\end{figure}

  \item $u$ with $c$.
  \begin{figure}[H]
  \centering \resizebox{0.9\textwidth}{!}{\includegraphics{Wexler13_7.png}}
  \caption{}\label{fig7}
\end{figure}


  \item $v_x$ (horizontal velocity) with $c$ for $\beta \gg \epsilon$.
  \begin{figure}[H]
  \centering \resizebox{0.9\textwidth}{!}{\includegraphics{Wexler13_6.png}}
  \caption{}\label{fig6}
\end{figure}


\end{itemize}
\end{itemize}


\clearpage


\section{Effect of the orientation of the harbor seal vibrissa based biomimetic cylinder on hydrodynamic forces and vortex induced frequency}

\textit{2017. AIP ADVANCES}

\subsection{Summary}
This study aims at finding the effect of the angle of attack (AOA) on flow characteristics around the harbor seal vibrissa shaped cylinder, to cover the change of flow direction during the harbor seal’s movements and surrounding conditions.
\begin{itemize}
\item
Parameters space variables: 
\begin{enumerate}
\item 
$AOA$: $0\leq AOA \leq 90$
\item 
$Re = 500$
\item 
Vortex shedding frequency
\item 
Shear layers
\end{enumerate}
\item
Computational domain: 
\begin{figure}[H]
  \centering \resizebox{0.9\textwidth}{!}{\includegraphics{Kim17_1.png}}
  \caption{}\label{fig1}
\end{figure}
\item
Results: 
\begin{itemize}
	\item
	The difference of force coefficients between the HSV and the elliptic cylinder can be classified into three regimes of one large variation region, two invariant regimes according to AOA.

	\item Elliptic cylinder showing the monotonically decrease of the vortex shedding frequency in AOA.

	\item HSV reveals the increasing and then decreasing behavior of the vortex shedding frequency along the AOA.

	\item The shear layers for the HSV is much longer than that of shear layers for the elliptic cylinder at low angles of the attack.

Note:

\end{itemize}
\end{itemize}

\subsection{Parameter definitions and non-dimensional groups}
To identify quantitatively the dependency of spanwise domain size, the time-averaged drag coefficient ($\overline{C_D}$), the Root mean square (RMS) value of lift fluctuation ($C_{L\,,RMS}$) and the Strouhal number ($St$) are compared
\begin{align*}
C_D &= \frac{2F_D}{\rho\,U_{\infty}^2\,A} \\ 
C_L &= \frac{2F_L}{\rho\,U_{\infty}^2\,A}  \\ 
St &= \frac{f_s\,D_h}{U_{\infty}}
\end{align*}

Note:

The time history of the lift force is generally used to estimate the vortex shedding frequency which is usually identified by the Strouhal number ($St$) as the dimensionless value.
\subsection{Results}
\begin{itemize}

	\item Comparison of the time histories of $C_D$ and $C_L$
	
\begin{figure}[H]
  \centering \resizebox{0.8\textwidth}{!}{\includegraphics{Kim17_22.png}}
  \caption{}\label{fig22}
\end{figure}

	\item Comparison of the $\overline{C_D}$ and $\overline{C_L}$
\begin{figure}[H]
  \centering \resizebox{0.8\textwidth}{!}{\includegraphics{Kim17_2.png}}
  \caption{}\label{fig2}
\end{figure}
	\item Power spectral density(PSD) of the time-dependent lift coefficient
\begin{figure}[H]
  \centering \resizebox{0.8\textwidth}{!}{\includegraphics{Kim17_3.png}}
  \caption{}\label{fig3}
\end{figure}
\begin{figure}[H]
  \centering \resizebox{0.8\textwidth}{!}{\includegraphics{Kim17_4.png}}
  \caption{}\label{fig4}
\end{figure}
	\item Contours of the time-averaged and instantaneous spanwise vorticity
\begin{figure}[H]
  \centering \resizebox{0.9\textwidth}{!}{\includegraphics{Kim17_5.png}}
  \caption{}\label{fig5}
\end{figure}
\begin{figure}[H]
  \centering \resizebox{0.9\textwidth}{!}{\includegraphics{Kim17_6.png}}
  \caption{}\label{fig6}
\end{figure}
	\item Contours of the turbulent kinetic energy (TKE)
\begin{figure}[H]
  \centering \resizebox{0.9\textwidth}{!}{\includegraphics{Kim17_10.png}}
  \caption{}\label{fig10}
\end{figure}
\begin{equation*}
k=\frac{1}{2}\big(\overline{(u^{\prime})^2}+\overline{(v^{\prime})^2}+\overline{(w^{\prime})^2})
\end{equation*}
	where $u^{\prime}$, $v^{\prime}$ and $w^{\prime}$ are the fluctuations in velocity in the $x$, $y$ and $z$ directions, respectively.

\end{itemize}

\clearpage

\section{Bristles reduce the force required to ‘fling’ wings apart in the smallest insects}

\textit{2016. Journal of Experimental Biology}

\subsection{Summary}
\begin{itemize}
\item
Gathered morphological data on 23 species bristles
\item 
Showed that the existence of bristles reduces fling while maintaining lift using numerical simulations
\item
Parameters space variables: 
\begin{enumerate}
\item 
Bristle-based $Re_{b}$: $10^{−3}\leq Re_b \leq 10^{−1}$
\item 
Spacing (The gap spacing to bristle diameter ratio; Fig.~\ref{fig2})
\begin{figure}[H]
  \centering \resizebox{0.25\textwidth}{!}{\includegraphics{Jones16_2.png}}
  \caption{}\label{fig2}
\end{figure}

\item 
Angle of attack
\item 
wing-wing interaction
\end{enumerate}
\item
Computational domain: 500 bristle diameters wide and high, and the bristles were at least $100$ bristle diameters from the edges of the computational domain (Regions of fluid that were close to the bristles or whose vorticity was above $0.125 s^{−1}$ were discretized at the highest level of refinement.)
\item
Numerical method: immersed boundary method
\item
Results: 
\begin{itemize}
\item Bristled wing experiences less force than a solid wing
\item Force reduction with increasing gap to diameter ratio is greater at higher angles of attack, therefore, bristled wings may act more like solid wings at lower angles of attack than they do at higher angles of attack 
\item In wing-wing interactions bristled wings significantly decrease the drag required to fling two wings apart compared with solid wings, especially at lower Reynolds numbers.

Note:

While previous work suggests that a single wing with bristles engaged in steady translation or rotation is almost as effective as a solid wing at producing aerodynamic forces, the bristles might offer an aerodynamic benefit during wing–wing interactions.
\end{itemize}
\end{itemize}

\subsection{Parameter definitions and non-dimensional groups}
The bristles on insect wings was modeled(approximated) as a row of cylinders (Fig.~\ref{fig1} A,B):

\begin{figure}[H]
  \centering \resizebox{0.6\textwidth}{!}{\includegraphics{Jones16_1.png}}
  \caption{}\label{fig1}
\end{figure}

\begin{itemize}
\item 
\textbf{Bristle-based $R_e$} (of the order $10^{-2}$):
\begin{equation*}
Re_b = \frac{\rho_{\mathrm{fluid}}\,U\,D}{\mu}
\end{equation*}
Note: chord-based Reynolds number ($Re_c$) for these insects is of the order of 10.
\item 
\textbf{Leakiness} (Fig.~\ref{fig1} C):

The ratio of the volume of viscous fluid that actually moves between a pair of bristles ($V_\mathrm{leak}$) to the volume across which that bristle pair sweeps ($V_{\mathrm{sweep}}$) in a unit of time (the volume of fluid that would move between the bristles in an inviscid fluid).


\item
\textbf{Dimensionless drag}
\begin{itemize}
	\item
Non-dimensionalized by $0.5\rho\,U^2\,L$, where $U$ is the characteristic steady-state velocity and $L$ is the characteristic length and depended on the application. 
	\item
For simulations with only two bristles, $L$ was the bristle diameter, $D$, and the reported force was that for each individual bristle. 
	\item
For simulations with full-length wings, $L$ was the chord length, $c$, and the reported force was the sum of the forces experienced by the entire row of bristles. $C_L$ and $C_D$ denote the lift and drag coefficients, respectively.
\end{itemize}
\end{itemize}

\subsection{Results}
\subsubsection{Bristles and angle of attack}

Note:

 Cheer et al. (2012) discovered that the speed and approaching angle of the flow play a role in generating vortices that reduce the effective size of the gap between rakers.

\clearpage

\section{Marine crustaceans with hairy appendages: Role of hydrodynamic boundary layers in sensing and feeding}

\textit{201p. Physical Review Fluids}
\subsection{Summary}
    
The equations of motion for fluids with $Re = O(1)$ are nonlinear and therefore difficult and costly to solve, there has been an emergence of asymptotic and numerical theories to enable rational design with inertia. However, in this regime, results are sensitive to variations in the boundary conditions, and so far theories have been ad-hoc. Inertial flow over complicated and intricate surfaces is an open field for study.

To close this gap between biology and rational design, we investigate a bio-inspired model system of rigid hairs subject to inertial flow at $Re = O(1)$. To use inertia as a design element, similar to the olfaction mechanism in crustaceans, we need a theory for the flow phase based on experimental parameters such as **hair length, diameter**, and **spacing length.

**Koehl et al. observed that crustaceans flick their antennae at different speeds, intentionally manipulating the Reynolds number to achieve different states of flow. 

The two states observed are called:
rake, where the **fluid inside the bed of hairs is stagnant**
, and 
sieve, where the **fluid inside the bed of hairs travels opposite the motion of the antenna**. Koehl et al. have conjectured that crustaceans use these different phases of flow to **aid olfaction (sense of smelling/tasting).**

Schematic:
\begin{figure}[H]
  \centering \resizebox{0.25\textwidth}{!}{\includegraphics{hood19_1.png}}
  \caption{}\label{fig2}
\end{figure}


Features of the flow are set by the smallest length scale in this system, which is the diameter of the hairs, $d_h = 1 mm$. We define the characteristic velocity to be the maximum flow speed $U$ in an undisturbed channel. 

Then, for water with density $\rho$ and viscosity $\mu$, we define the Reynolds number to be: $Re = \frac{\rho U d_h}{\mu}$. 

**We measure the flow phase as a function of $Re$ and the separation lengths $\delta$ of the hair bed by measuring the magnitude of the flow velocity in the center of the bed.

\begin{figure}[H]
  \centering \resizebox{0.25\textwidth}{!}{\includegraphics{Hood19_2.png}}
  \caption{}\label{fig2}
\end{figure}

<img src="../images/hood19-2.png" alt="Drawing" style="width: 650px;"/>

<big>Goal: Design a predictive thepry that Given $Re$ and $\delta$ predicts the exhibited flow phase.</big>

<img src="../images/hood19-3.png" alt="Drawing" style="width: 650px;"/>
<img src="../images/hood19-4.png" alt="Drawing" style="width: 650px;"/>
<img src="../images/hood19-5.png" alt="Drawing" style="width: 650px;"/>

To determine the depth of the boundary layer, we consider the $p = −0.1$ level set (The boundary that encloses the region where $\vert\frac{w'}{U}\vert>0.1$)
then measure the critical radius ($r_c$) vs. $Re$ (Fig. 3)
Fit the resulting curve (Fig. 3)
Measure the disturbance flow velocity $w'$ (Fig. 4)

Defined measurables (flow is in $z$ direction):
    
$\frac{\langle w \rangle}{U}$ vs. $Re$
$\frac{\delta}{L_h}$ vs. $Re$
$\theta$ vs. $Re$

\clearpage

Link:\href{https://www.cambridge.org/core/services/aop-cambridge-core/content/view/16FACE9E75678D170A4ECCC7990C8158/S0022112096004673a.pdf/experiments_on_dragreducing_surfaces_and_their_optimization_with_an_adjustable_geometry.pdf}{Link to the article}
Summary: Geometrical and mechanical optimization of riblet network for drag reduction. A (mostly) experimental study with a lot of data. 

Relevant sections: The entire work is relevant but more specifically:
\begin{itemize}
	\item Chapter 6
	\item Chapter 7
\end{itemize}

If we decide on doing a parametric study, these two chapters can give us some ideas on what parameters to play with.

\end{section}

Also:

Link:\href{https://link.springer.com/content/pdf/10.1007/s003480000150.pdf}{Flow field analysis of a turbulent boundary layer over a riblet surface}

Link:\href{https://link.springer.com/content/pdf/10.1007%2Fs001140050696.pdf}{Fluid Mechanics of Biological Surfaces and their Technological Application}

\begin{section}{Surface structure and dimensional effects on the aerodynamics of an owl-based wing model}

Link:\href{https://pubs.acs.org/doi/pdf/10.1021/la2043729}{Link to the article}
Summary: characterizing flow field stability by comparing the aerodynamics of two hairy surface and a clean surface. 

I think this is a good reference if we decide on quantifying the vortex shedding phenomena (Figure 14)

\end{section}

\begin{section}{Interaction between hairy surfaces and turbulence for different surface time scales}

Link:\href{https://www.cambridge.org/core/services/aop-cambridge-core/content/view/3EF8797679D529AA393F7D41FCD93B70/S0022112018009357a.pdf/interaction_between_hairy_surfaces_and_turbulence_for_different_surface_time_scales.pdf}{Link to the article}


Summary: surface time scale analysis using numerical simulations of a bed of filaments in a turbulent channel flow.

From the abstract: "filamentous bed of a given geometry can modify a turbulent flow very differently depending on the resonance frequency of the surface, which is determined by the elasticity and mass of the filaments." 

Eventhough their focus is on the turbulence flow, I think this is a very important reference. They define/introduce some important non-dimensional groups which can give us some hints on what parameters to modify/study. I will need to read this work in more details.

Also:

Link:\href{https://www.cambridge.org/core/services/aop-cambridge-core/content/view/D57BDEAB405629B48AB60FCD40A1D2B9/S0022112009006119a.pdf/passive_separation_control_using_a_selfadaptive_hairy_coating.pdf}{Passive separation control using a self-adaptive hairy coating}
\end{section}

\begin{section}{Bioinspired Structured Surfaces}

Link:\href{https://pubs.acs.org/doi/pdf/10.1021/la2043729}{Link to the article}
Summary: An overview of four topics: Lotus effect, rose petal effect, gecko feet, and shark skin.

Relevant sections:
\begin{itemize}
	\item Chapter 5 (SHARK SKIN)
	\item Figure 15
	\item Figure 16
\end{itemize}
\end{section}


\begin{section}{Bioinspired Structured Surfaces}

Link:\href{https://pubs.acs.org/doi/pdf/10.1021/la2043729}{Link to the article}
Summary: An overview of four topics: Lotus effect, rose petal effect, gecko feet, and shark skin.

Relevant sections:
\begin{itemize}
	\item Chapter 5 (SHARK SKIN)
	\item Figure 15
	\item Figure 16
\end{itemize}
\end{section}
\end{document}
