\documentclass[preprint, letterpaper, nobibnotes, aps, superscriptaddress,prb]{revtex4-1}

\usepackage{amsmath,amssymb} 
\usepackage{bm}
\usepackage{hyperref}
\usepackage{subfigure}
\usepackage{graphicx}
\usepackage{color}
\usepackage{float}
\usepackage{multirow}
\usepackage{etoolbox}
\usepackage{booktabs}
\usepackage{ltablex}
\usepackage{xr}
\begin{document}

\section{Effect of the orientation of the harbor seal vibrissa based biomimetic cylinder on hydrodynamic forces and vortex induced frequency}

\textit{2017. AIP ADVANCES}

\subsection{Summary}
This study aims at finding the effect of the angle of attack (AOA) on flow characteristics around the harbor seal vibrissa shaped cylinder, to cover the change of flow direction during the harbor seal’s movements and surrounding conditions.
\begin{itemize}
\item
Parameters space variables: 
\begin{enumerate}
\item 
$AOA$: $0\leq AOA \leq 90$
\item 
$Re = 500$
\item 
Vortex shedding frequency
\item 
Shear layers
\end{enumerate}
\item
Computational domain: 
\begin{figure}[H]
  \centering \resizebox{0.9\textwidth}{!}{\includegraphics{Kim17_1.png}}
  \caption{}\label{fig1}
\end{figure}
\item
Results: 
\begin{itemize}
	\item
	The difference of force coefficients between the HSV and the elliptic cylinder can be classified into three regimes of one large variation region, two invariant regimes according to AOA.

	\item Elliptic cylinder showing the monotonically decrease of the vortex shedding frequency in AOA.

	\item HSV reveals the increasing and then decreasing behavior of the vortex shedding frequency along the AOA.

	\item The shear layers for the HSV is much longer than that of shear layers for the elliptic cylinder at low angles of the attack.

Note:

\end{itemize}
\end{itemize}

\subsection{Parameter definitions and non-dimensional groups}
To identify quantitatively the dependency of spanwise domain size, the time-averaged drag coefficient ($\overline{C_D}$), the Root mean square (RMS) value of lift fluctuation ($C_{L\,,RMS}$) and the Strouhal number ($St$) are compared
\begin{align*}
C_D &= \frac{2F_D}{\rho\,U_{\infty}^2\,A} \\ 
C_L &= \frac{2F_L}{\rho\,U_{\infty}^2\,A}  \\ 
St &= \frac{f_s\,D_h}{U_{\infty}}
\end{align*}

Note:

The time history of the lift force is generally used to estimate the vortex shedding frequency which is usually identified by the Strouhal number ($St$) as the dimensionless value.
\subsection{Results}
\begin{itemize}

	\item Comparison of the time histories of $C_D$ and $C_L$
	
\begin{figure}[H]
  \centering \resizebox{0.8\textwidth}{!}{\includegraphics{Kim17_22.png}}
  \caption{}\label{fig22}
\end{figure}

	\item Comparison of the $\overline{C_D}$ and $\overline{C_L}$
\begin{figure}[H]
  \centering \resizebox{0.8\textwidth}{!}{\includegraphics{Kim17_2.png}}
  \caption{}\label{fig2}
\end{figure}
	\item Power spectral density(PSD) of the time-dependent lift coefficient
\begin{figure}[H]
  \centering \resizebox{0.8\textwidth}{!}{\includegraphics{Kim17_3.png}}
  \caption{}\label{fig3}
\end{figure}
\begin{figure}[H]
  \centering \resizebox{0.8\textwidth}{!}{\includegraphics{Kim17_4.png}}
  \caption{}\label{fig4}
\end{figure}
	\item Contours of the time-averaged and instantaneous spanwise vorticity
\begin{figure}[H]
  \centering \resizebox{0.9\textwidth}{!}{\includegraphics{Kim17_5.png}}
  \caption{}\label{fig5}
\end{figure}
\begin{figure}[H]
  \centering \resizebox{0.9\textwidth}{!}{\includegraphics{Kim17_6.png}}
  \caption{}\label{fig6}
\end{figure}
	\item Contours of the turbulent kinetic energy (TKE)
\begin{figure}[H]
  \centering \resizebox{0.9\textwidth}{!}{\includegraphics{Kim17_10.png}}
  \caption{}\label{fig10}
\end{figure}
\begin{equation*}
k=\frac{1}{2}\big(\overline{(u^{\prime})^2}+\overline{(v^{\prime})^2}+\overline{(w^{\prime})^2})
\end{equation*}
	where $u^{\prime}$, $v^{\prime}$ and $w^{\prime}$ are the fluctuations in velocity in the $x$, $y$ and $z$ directions, respectively.

\end{itemize}

\end{document}
